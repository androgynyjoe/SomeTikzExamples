\documentclass{report}

%basic packages that most documents have
\usepackage[utf8]{inputenc}
\usepackage[T1]{fontenc}
\usepackage{amsmath}
\usepackage{amssymb}
\usepackage{amsthm}
\usepackage{mathtools}

%packages specific to this document
\usepackage{siunitx}
\usepackage[svgnames]{xcolor}
\usepackage{tikz}
\usepackage{pgfplots}
\pgfplotsset{compat=1.12}
\usetikzlibrary{arrows,patterns,calc,matrix,decorations.markings}
\usepackage{caption}
\captionsetup{format=hang,font=small,justification=raggedright}
\captionsetup[subfloat]{font=footnotesize}
\usepackage{subfig}
\usepackage[colorlinks=true, allcolors=black]{hyperref}
\usepackage[noabbrev]{cleveref}
\usepackage{autonum}
\crefname{figure}{figure}{figures}
\crefformat{equation}{equation~\textup{#2#1#3}}
\Crefformat{equation}{Equation~\textup{#2#1#3}}

%COLORS USED
\colorlet{ConeColor}{DarkBlue}
\colorlet{AxisColor}{red}
\colorlet{PlaneColor}{DarkGreen}
\colorlet{HighlightColor}{red}
\colorlet{ParabolaColor}{DarkBlue}
\colorlet{FocDirColor}{DarkBlue}
\colorlet{BeamColor}{orange}

%THEOREM ENVIRONMENTS
\theoremstyle{definition}
\newtheorem{HistoricalNote}{Historical Note}
\newtheorem{Definition}{Definition}
\newtheorem{Theorem}{Theorem}
\newtheorem{Remark}{Remark}
\newtheorem{YouTryThis}{You Try This}
\newtheorem{TeachingConnection}{Teaching Connection}

%MY CUSTOM STUFF
\tikzstyle directed=[postaction={decorate,decoration={markings, mark=at position .5 with {\arrow[scale=1]{stealth}}}}]
\tikzstyle reverse directed=[postaction={decorate,decoration={markings, mark=at position .5 with {\arrowreversed[scale=1]{stealth};}}}]

\begin{document}

\begin{figure}[ht]
\centering%
\subfloat[A right circular cone.]{%
\begin{tikzpicture}[x = (270:0.25cm), y = (0:1cm), z = (90:1cm), scale=0.9]
\def\numSamples{400}
\def\circRad{2.5}
\def\circZ{4}
\def\generatorAng{140}
\def\pointSep{1}

%Draws the Cone
\foreach \t in {1,...,\numSamples} {
\fill[ConeColor!45, opacity=0.25] (0,0,0) -- ({\circRad*cos((\t*2*pi/\numSamples) r)},{\circRad*sin((\t*2*pi/\numSamples) r)},-\circZ) -- ({\circRad*cos(((\t-1)*2*pi/\numSamples) r)},{\circRad*sin(((\t-1)*2*pi/\numSamples) r)},-\circZ);}
\draw[ConeColor] (0,0,-\circZ) circle(\circRad);

%Draws the Vertex
\draw (0,0,0) node[AxisColor, circle, fill=AxisColor, inner sep=\pointSep]{};
\draw[AxisColor] (0,0,0) node[right]{$V$};

%Draw the Angle
\draw[gray,thin] (0,0,-\circZ)--(0,\circRad,-\circZ);
\draw[gray,thin] (0,0.35,-\circZ)--(0,0.35,{0.35-\circZ})--(0,0,{0.35-\circZ});

%Draws the Origin
\draw (0,0,-\circZ) node[AxisColor, circle, fill=AxisColor, inner sep=\pointSep]{};
\draw[AxisColor] (0,0,-\circZ) node[left]{$O$};

%Draws the Axis
\draw[AxisColor] (0,0,0) -- (0,0,-\circZ);
\end{tikzpicture}
\label{fig:cone1}}\qquad
\subfloat[A (non-right) circular cone.]{%
\begin{tikzpicture}[x = (270:0.25cm), y = (0:1cm), z = (90:1cm), scale=0.9]
\def\numSamples{400}
\def\circRad{2.5}
\def\circZ{4}
\def\generatorAng{140}
\def\pointSep{1}

%Draws the Cone
\foreach \t in {1,...,\numSamples} {
\fill[ConeColor!45, opacity=0.25] (0,1,0) -- ({\circRad*cos((\t*2*pi/\numSamples) r)},{\circRad*sin((\t*2*pi/\numSamples) r)},-\circZ) -- ({\circRad*cos(((\t-1)*2*pi/\numSamples) r)},{\circRad*sin(((\t-1)*2*pi/\numSamples) r)},-\circZ);}
\draw[ConeColor] (0,0,-\circZ) circle(\circRad);

%Draws the Vertex
\draw (0,1,0) node[AxisColor, circle, fill=AxisColor, inner sep=\pointSep]{};
\draw[AxisColor] (0,1,0) node[right]{$V$};

%Draws the Origin
\draw (0,0,-\circZ) node[AxisColor, circle, fill=AxisColor, inner sep=\pointSep]{};
\draw[AxisColor] (0,0,-\circZ) node[left]{$O$};

%Draws the Axis
\draw[AxisColor] (0,1,0) -- (0,0,-\circZ);
\end{tikzpicture}
\label{fig:cone2}}
\caption{Examples of circular cones.}
\label{fig:cone}
\end{figure}

\begin{figure}[ht]
\centering%
\begin{tikzpicture}[x = (210:0.3cm), y = (0:1cm), z = (90:1cm), scale=0.6]
\def\numSamples{400}
\def\circRad{4}
\def\circZ{4}
\def\generatorAng{140}
\def\pointSep{1}

%Draws the Vertical Axis
\draw[gray,<->] (0,0,{\circZ+0.5})node[above]{\scriptsize{$y$}} -- (0,0,{-\circZ-0.5});
\draw[gray,<->] ({\circZ+0.5},0,0)node[below left]{\scriptsize{$z$}} -- ({-\circZ-0.5},0,0);
\draw[gray,<->] (0,{\circZ+0.5},0)node[right]{\scriptsize{$x$}} -- (0,{-\circZ-0.5},0);

%Draws the Cone
\foreach \t in {1,...,\numSamples} {
\fill[ConeColor!45, opacity=0.25] (0,0,0) -- ({\circRad*cos((\t*2*pi/\numSamples) r)},{\circRad*sin((\t*2*pi/\numSamples) r)},\circZ) -- ({\circRad*cos(((\t-1)*2*pi/\numSamples) r)},{\circRad*sin(((\t-1)*2*pi/\numSamples) r)},\circZ);
\fill[ConeColor!45, opacity=0.25] (0,0,0) -- ({\circRad*cos((\t*2*pi/\numSamples) r)},{\circRad*sin((\t*2*pi/\numSamples) r)},-\circZ) -- ({\circRad*cos(((\t-1)*2*pi/\numSamples) r)},{\circRad*sin(((\t-1)*2*pi/\numSamples) r)},-\circZ);}
\draw[ConeColor!45, thin, dashed] (0,0,\circZ) circle(\circRad);
\draw[ConeColor!45, thin, dashed] (0,0,-\circZ) circle(\circRad);

%Draws the Origin
\draw (0,0,0) node[AxisColor, circle, fill=AxisColor, inner sep=\pointSep]{};
\draw[AxisColor] (0,0,0) node[right]{$O$};

%Draws the Generator
\draw[AxisColor,<->] ({-1.1*\circRad*sin(\generatorAng)},{1.1*\circRad*cos(\generatorAng)},{1.1*\circZ}) -- ({1.1*\circRad*sin(\generatorAng)},{-1.1*\circRad*cos(\generatorAng)},{-1.1*\circZ});

%Draws P
\draw ({-\circRad*sin(\generatorAng)},{\circRad*cos(\generatorAng)},\circZ) node[AxisColor, circle, fill=AxisColor, inner sep=\pointSep]{};
\draw[AxisColor] ({-\circRad*sin(\generatorAng)},{\circRad*cos(\generatorAng)},{\circZ}) node[left]{$P$};

%Draws P^\prime
\draw ({\circRad*sin(\generatorAng)},{-\circRad*cos(\generatorAng)},-\circZ) node[AxisColor, circle, fill=AxisColor, inner sep=\pointSep]{};
\draw[AxisColor] ({\circRad*sin(\generatorAng)},{-\circRad*cos(\generatorAng)},{-\circZ}) node[right]{$P^\prime$};

%Draws \alpha
\draw[AxisColor] plot[samples=25, domain={atan(\circZ/\circRad)}:90, variable=\t] ({-sqrt(\circZ*\circZ + \circRad*\circRad)*cos(\t)/6 * sin(\generatorAng)} , {sqrt(\circZ*\circZ + \circRad*\circRad)*cos(\t)/6 * cos(\generatorAng)} , {sqrt(\circZ*\circZ + \circRad*\circRad)*sin(\t)/6});
\draw[AxisColor] ({-sqrt(\circZ*\circZ + \circRad*\circRad)*cos(atan(\circZ/\circRad)/2 + 45)/4.5 * sin(\generatorAng)} , {sqrt(\circZ*\circZ + \circRad*\circRad)*cos(atan(\circZ/\circRad)/2 + 45)/4.5 * cos(\generatorAng)} , {sqrt(\circZ*\circZ + \circRad*\circRad)*sin(atan(\circZ/\circRad)/2 + 45)/4.5}) node{$\alpha$};
\end{tikzpicture}
\caption{A right double infinite cone.}
\label{fig:infinitecone}
\end{figure}

\begin{figure}[ht]
\centering%
\subfloat[Circle]{%
\begin{tikzpicture}[x = (225:0.3cm), y = (0:1cm), z = (90:1cm), scale=0.5]
\def\numSamples{400}
\def\circRad{4}
\def\circZ{4}
\def\generatorAng{140}
\def\pointSep{1.5}

%Draws the Vertical Axis
\draw[gray] (0,0,\circZ) -- (0,0,-\circZ);

%Draws the Cone
\foreach \t in {1,...,\numSamples} {
\fill[ConeColor!45, opacity=0.25] (0,0,0) -- ({\circRad*cos((\t*2*pi/\numSamples) r)},{\circRad*sin((\t*2*pi/\numSamples) r)},\circZ) -- ({\circRad*cos(((\t-1)*2*pi/\numSamples) r)},{\circRad*sin(((\t-1)*2*pi/\numSamples) r)},\circZ);
\fill[ConeColor!45, opacity=0.25] (0,0,0) -- ({\circRad*cos((\t*2*pi/\numSamples) r)},{\circRad*sin((\t*2*pi/\numSamples) r)},-\circZ) -- ({\circRad*cos(((\t-1)*2*pi/\numSamples) r)},{\circRad*sin(((\t-1)*2*pi/\numSamples) r)},-\circZ);}
\draw[ConeColor!45, thin, dashed] (0,0,\circZ) circle(\circRad);
\draw[ConeColor!45, thin, dashed] (0,0,-\circZ) circle(\circRad);

%Draws the Plane
\draw[PlaneColor!45, fill=PlaneColor!45, opacity=.35] (4,4,2)--(4,-4,2)--(-4,-4,2)--(-4,4,2)--(4,4,2);

%Draws the Conic
\draw[AxisColor] (0,0,2) circle(2);
\end{tikzpicture}
\label{fig:conics_circle}}\qquad
\subfloat[Ellipse]{%
\begin{tikzpicture}[x = (225:0.3cm), y = (0:1cm), z = (90:1cm), scale=0.5]
\def\numSamples{400}
\def\circRad{4}
\def\circZ{4}
\def\generatorAng{140}
\def\pointSep{1.5}

%Draws the Vertical Axis
\draw[gray] (0,0,\circZ) -- (0,0,-\circZ);

%Draws the Cone
\foreach \t in {1,...,\numSamples} {
\fill[ConeColor!45, opacity=0.25] (0,0,0) -- ({\circRad*cos((\t*2*pi/\numSamples) r)},{\circRad*sin((\t*2*pi/\numSamples) r)},\circZ) -- ({\circRad*cos(((\t-1)*2*pi/\numSamples) r)},{\circRad*sin(((\t-1)*2*pi/\numSamples) r)},\circZ);
\fill[ConeColor!45, opacity=0.25] (0,0,0) -- ({\circRad*cos((\t*2*pi/\numSamples) r)},{\circRad*sin((\t*2*pi/\numSamples) r)},-\circZ) -- ({\circRad*cos(((\t-1)*2*pi/\numSamples) r)},{\circRad*sin(((\t-1)*2*pi/\numSamples) r)},-\circZ);}
\draw[ConeColor!45, thin, dashed] (0,0,\circZ) circle(\circRad);
\draw[ConeColor!45, thin, dashed] (0,0,-\circZ) circle(\circRad);

%Draws the Plane
\draw[PlaneColor!45, fill=PlaneColor!45, opacity=.35] (4,4,{1.5 + 4*0 + 4*(-0.5)})--(4,-4,{1.5 + 4*0 - 4*(-0.5)})--(-4,-4,{1.5 - 4*0 - 4*(-0.5)})--(-4,4,{1.5 - 4*0 + 4*(-0.5)})--(4,4,{1.5 + 4*0 + 4*(-0.5)});

%Draws the Conic
\draw[AxisColor] plot[domain={-3}:{1}, samples=100, variable=\y] ({sqrt(9/4 - (3/2)*\y - (3/4)*\y*\y)} , {\y} , {1.5 - \y/2});
\draw[AxisColor] plot[domain={-3}:{1}, samples=100, variable=\y] ({-sqrt(9/4 - (3/2)*\y - (3/4)*\y*\y)} , {\y} , {1.5 - \y/2});
\end{tikzpicture}
\label{fig:conics_ellipse}}\\
\subfloat[Parabola]{%
\begin{tikzpicture}[x = (225:0.3cm), y = (0:1cm), z = (90:1cm), scale=0.5]
\def\numSamples{400}
\def\circRad{4}
\def\circZ{4}
\def\generatorAng{140}
\def\pointSep{1.5}

%Draws the Vertical Axis
\draw[gray] (0,0,\circZ) -- (0,0,-\circZ);

%Draws the Cone
\foreach \t in {1,...,\numSamples} {
\fill[ConeColor!45, opacity=0.25] (0,0,0) -- ({\circRad*cos((\t*2*pi/\numSamples) r)},{\circRad*sin((\t*2*pi/\numSamples) r)},\circZ) -- ({\circRad*cos(((\t-1)*2*pi/\numSamples) r)},{\circRad*sin(((\t-1)*2*pi/\numSamples) r)},\circZ);
\fill[ConeColor!45, opacity=0.25] (0,0,0) -- ({\circRad*cos((\t*2*pi/\numSamples) r)},{\circRad*sin((\t*2*pi/\numSamples) r)},-\circZ) -- ({\circRad*cos(((\t-1)*2*pi/\numSamples) r)},{\circRad*sin(((\t-1)*2*pi/\numSamples) r)},-\circZ);}
\draw[ConeColor!45, thin, dashed] (0,0,\circZ) circle(\circRad);
\draw[ConeColor!45, thin, dashed] (0,0,-\circZ) circle(\circRad);

%Draws the Plane
\draw[PlaneColor!45, fill=PlaneColor!45, opacity=.35] (4,4,{(1) + 4*(0) + 4*(-1)})--(4,-3,{(1) + 4*(0) - 3*(-1)})--(-4,-3,{(1) - 4*(0) - 3*(-1)})--(-4,4,{(1) - 4*(0) + 4*(-1)})--(4,4,{(1) + 4*(0) + 4*(-1)});

%Draws the Conic
\draw[AxisColor] plot[domain={-2.64}:{2.64}, samples=100, variable=\x] ({\x} , {-0.5*\x*\x + 0.5} , {0.5*\x*\x + 0.5});
\end{tikzpicture}
\label{fig:conics_parabola}}\qquad
\subfloat[Hyperbola]{%
\begin{tikzpicture}[x = (225:0.3cm), y = (0:1cm), z = (90:1cm), scale=0.5]
\def\numSamples{400}
\def\circRad{4}
\def\circZ{4}
\def\generatorAng{140}
\def\pointSep{1.5}

%Draws the Vertical Axis
\draw[gray] (0,0,\circZ) -- (0,0,-\circZ);

%Draws the Cone
\foreach \t in {1,...,\numSamples} {
\fill[ConeColor!45, opacity=0.25] (0,0,0) -- ({\circRad*cos((\t*2*pi/\numSamples) r)},{\circRad*sin((\t*2*pi/\numSamples) r)},\circZ) -- ({\circRad*cos(((\t-1)*2*pi/\numSamples) r)},{\circRad*sin(((\t-1)*2*pi/\numSamples) r)},\circZ);
\fill[ConeColor!45, opacity=0.25] (0,0,0) -- ({\circRad*cos((\t*2*pi/\numSamples) r)},{\circRad*sin((\t*2*pi/\numSamples) r)},-\circZ) -- ({\circRad*cos(((\t-1)*2*pi/\numSamples) r)},{\circRad*sin(((\t-1)*2*pi/\numSamples) r)},-\circZ);}
\draw[ConeColor!45, thin, dashed] (0,0,\circZ) circle(\circRad);
\draw[ConeColor!45, thin, dashed] (0,0,-\circZ) circle(\circRad);

%Draws the Plane
\draw[PlaneColor!45, fill=PlaneColor!45, opacity=.35] (4,1.5,4)--(-4,1.5,4)--(-4,1.5,-4)--(4,1.5,-4)--(4,1.5,4);

%Draws the Conic
\draw[AxisColor] plot[domain=-3.708:3.708, samples=100, variable=\x] (\x, 1.5, {sqrt(pow(\x,2) + pow(1.5,2))});
\draw[AxisColor] plot[domain=-3.708:3.708, samples=100, variable=\x] (\x, 1.5, {-sqrt(pow(\x,2) + pow(1.5,2))});
\end{tikzpicture}
\label{fig:conics_hyperbola}}
\caption{Demonstrations of the various conic sections.}
\label{fig:conics}
\end{figure}

\begin{figure}[ht]
\centering%
\subfloat[A parabola in a plane defined by a focus and directrix.]{%
\scriptsize
\begin{tikzpicture}[scale=0.7]
\draw[white, fill=white] (3,-1) circle(0.06) node[below]{$B(x,-p)$};
\draw[dashed,thin] (0,-1.5)--(0,4);
\draw[thin] (.2,-1)--(.2,-.8)--(0,-.8);
\draw[FocDirColor, fill=FocDirColor] (0,1) circle(0.06) node[left]{$F$};
\draw[FocDirColor, <->] (-4.1,-1)--(4.1,-1) node[right]{$d$};
\draw[ParabolaColor, <->, thick] plot[domain=-4:4, samples=100, variable=\x] ({\x},{0.25*pow(\x,2)});
\draw[ParabolaColor, fill=ParabolaColor] (0,0) circle(0.06) node[below left]{$V$};
\end{tikzpicture}
\label{fig:parabola1}}\qquad
\subfloat[The same parabola with coordinates.]{%
\scriptsize
\begin{tikzpicture}[scale=0.7]
\draw[<->] (0,-1.5)--(0,4)node[above]{$y$};
\draw[<->] (-4.1,0)--(4.1,0)node[right]{$x$};
\draw[thin] (.2,-1)--(.2,-.8)--(0,-.8);
\draw[FocDirColor, fill=FocDirColor] (0,1) circle(0.06) node[left]{$F(0,p)$};
\draw[HighlightColor, fill=HighlightColor] (0,-1) circle(0.06) node[below left]{$(0,-p)$};
\draw[FocDirColor, <->] (-4.1,-1)--(4.1,-1) node[right]{$d$};
\draw[ParabolaColor, <->, thick] plot[domain=-4:4, samples=100, variable=\x] ({\x},{0.25*pow(\x,2)});
\draw[ParabolaColor, fill=ParabolaColor] (0,0) circle(0.06) node[below left]{$V(0,0)$};
\draw[HighlightColor, fill=HighlightColor] (3,2.25) circle(0.06) node[right]{$A(x,y)$};
\draw[HighlightColor, fill=HighlightColor] (3,-1) circle(0.06) node[below]{$B(x,-p)$};
\draw[HighlightColor] (0,1)--(3,2.25);
\draw[HighlightColor] (3,2.25)--(3,-1);
\end{tikzpicture}
\label{fig:parabola2}}
\caption{The graph of a parabola.  The focus is $F$ and the directrix is $y=d$.}
\label{fig:parabola}
\end{figure}

\begin{figure}[ht]
\centering%
\scriptsize
\begin{tikzpicture}[scale=0.7]
\draw[<->, ParabolaColor, thick] plot[domain=-4:4, samples=200] ({\x},{0.25*pow(\x,2)})node[above]{parabola};
\draw[<->,thin] plot[domain=0:4, samples=2] ({\x},{\x-1}) node[right]{tangent line};
\draw[<->,thin] plot[domain=0:4, samples=2] ({\x},{-\x+3}) node[right]{normal line};
\draw[thin] (2.2,0.8)--(2.4,1)--(2.2,1.2);
\draw[directed, BeamColor, thick] plot[domain=-.5:2, samples=2] ({\x},{-0.25*(\x-2)+1});
\draw[BeamColor] (-.5,{-0.25*(-.5-2)+1}) node[left]{incoming};
\draw[reverse directed, BeamColor, thick] plot[domain={2-2.5/4}:2, samples=2] ({\x},{-4*(\x-2)+1});
\draw[BeamColor] ({2-2.5/4},{-4*(2-2.5/4-2)+1}) node[left]{outgoing};
\draw[HighlightColor] ({2+.75*cos(45)},{1+.75*sin(45)}) arc(45:{90+atan(1/4)}:.75);
\draw[HighlightColor] ({2+.75*cos(180-atan(1/4))},{1+.75*sin(180-atan(1/4))}) arc({180-atan(1/4)}:225:.75);
\draw[HighlightColor] ({2+1*cos((45+90+atan(1/4))/2)},{1+1*sin((45+90+atan(1/4))/2)}) node{$\beta$};
\draw[HighlightColor] ({2+1*cos((180-atan(1/4)+225)/2)},{1+1*sin((180-atan(1/4)+225)/2)}) node{$\alpha$};
\end{tikzpicture}
\caption{A beam of light being reflected by a parabolic mirror.}
\label{fig:beam}
\end{figure}

\begin{figure}[ht]
\centering%
\scriptsize
\begin{tikzpicture}[scale=0.7]
\draw[<->] (-4.1,0)--(4.1,0) node[right]{$x$};
\draw[<->] (0,-2.6)--(0,4.1) node[above]{$y$};
\draw[<->, FocDirColor] (-4.1,-1) node[left]{$d$}--(4.1,-1);
\draw[<->, ParabolaColor, thick] plot[domain=-4:4, samples=200] ({\x},{0.25*pow(\x,2)})node[above]{parabola};
\draw[<->,thin] plot[domain=-.25:4, samples=2] ({\x},{1.5*(\x-3) + 2.25}) node[right]{tangent line};
\draw[reverse directed, BeamColor, thick] (0,1)--(3,2.25);
\draw[reverse directed, BeamColor, thick] (3,2.25)--(3,4);
\draw[HighlightColor] (3,2.25)--(3,-1);
\draw[HighlightColor, fill=HighlightColor] (3,2.25) circle(0.06) node[right]{$A(x_0,y_0)$};
\draw[HighlightColor, fill=HighlightColor] (0,-2.25) circle(0.06) node[below right]{$B$};
\draw[HighlightColor, fill=HighlightColor] (3,0) circle(0.06) node[below right]{$D(x_0,0)$};
\draw[HighlightColor, fill=HighlightColor] (3,-1) circle(0.06) node[below right]{$C(x_0,-p)$};
\draw[FocDirColor, fill=FocDirColor] (0,1) circle(0.06) node[left]{$F(0,p)$};
\draw[HighlightColor, fill=HighlightColor] (0,-1) circle(0.06) node[below left]{$(0,-p)$};
\draw[HighlightColor, fill=HighlightColor] (3,3.8) circle(0.06) node[left]{$G$};
\draw[HighlightColor, fill=HighlightColor] (0,0) circle(0.06) node[below left]{$O$};
\draw[HighlightColor] (0,-1.5) arc(90:atan(1.5):.75);
\draw[HighlightColor] ({0+0.95*cos((90+atan(1.5))/2)},{-2.25+0.95*sin((90+atan(1.5))/2)}) node{$\alpha_2$};
\draw[HighlightColor] ({3},{2.25-0.75}) arc(270:{180+atan(1.5)}:.75);
\draw[HighlightColor] ({3+0.95*cos((270+180+atan(1.5))/2)},{2.25+0.95*sin((270+180+atan(1.5))/2)}) node{$\alpha_1$};
\draw[HighlightColor] ({3+0.6*cos((180+atan(1.5)))},{2.25+0.6*sin((180+atan(1.5)))}) arc({180+atan(1.5)}:{180+atan(1.25/3)}:.6);
\draw[HighlightColor] ({3+0.9*cos((180*2+atan(1.5)+atan(1.25/3))/2)},{2.25+0.9*sin((180*2+atan(1.5)+atan(1.25/3))/2)}) node{$\beta$};
\draw[HighlightColor] ({3+0.6*cos((atan(1.5)))},{2.25+0.6*sin((atan(1.5)))}) arc({atan(1.5)}:{90}:.6);
\draw[HighlightColor] ({3+0.9*cos((90+atan(1.5))/2)},{2.25+0.9*sin((90+atan(1.5))/2)}) node{$\alpha$};
\end{tikzpicture}
\caption{The figure used in the proof of the reflective property of the parabola.}
\label{fig:proof}
\end{figure}

\begin{figure}[ht]
\centering
\begin{tikzpicture}[x = (210:1cm), y = (0:1cm), z = (90:1cm), scale=.9]
%These parameters control the first plane.  What eventually gets plotted is the
%plane whose equation is z = \cp + (\xp)*x + (\yp)*y
\def\cp{2}
\def\xp{-0.25}
\def\yp{0.25}

%These parameters control the second plane.  What eventually gets plotted is the
%plane whose equation is z = \cq + (\xq)*x + (\yq)*y
\def\cq{3.5}
\def\xq{-0.333333}
\def\yq{-0.5}

%These will control the size of the axes.  
%IMPORTANT NOTE: The sizes of the x-axis and y-axis determine how much of each
%plane gets printed.  The planes are plotted with a function whose domain is the
%subset of the xy-axis [\xmin,\xmax]x[\ymin,\ymax].  As a result, changing the
%dimensions of the x-axis and y-axis will change a lot in the diagram.  However,
%the dimensions of the z-axis will change nothing.  The height of the planes is
%NOT determined dynamically based on the size of the z-axis.  The parameters
%\zmax and \zmin do nothing except determine the height of the z-axis.
\def\xmin{-1}
\def\xmax{5}
\def\ymin{-1}
\def\ymax{5}
\def\zmin{-1}
\def\zmax{5}
\def\axisextra{.5}

%This is the color that gets used for the planes.  Note that gray works fine.
\colorlet{planecolor1}{blue}
\colorlet{planecolor2}{red}

%Plot the axes and their labels
\draw[<->,help lines] ({\xmin-\axisextra},0,0) -- ({\xmax+\axisextra},0,0);
\draw[<->,help lines] (0,{\ymin-\axisextra},0) -- (0,{\ymax+\axisextra},0);
\draw[<->,help lines] (0,0,{\zmin-\axisextra}) -- (0,0,{\zmax+\axisextra});
\draw ({\xmax+1.5*\axisextra},0,0) node{\tiny{$x$}};
\draw (0,{\ymax+1.5*\axisextra},0) node{\tiny{$y$}};
\draw (0,0,{\zmax+1.5*\axisextra}) node{\tiny{$z$}};

%Plot z = \cp + (\xp)*x + (\yp)*y
\draw[planecolor1!45!black, fill=planecolor1!45, fill opacity=0.25] (\xmin,\ymin,{\cp+\xp*\xmin+\yp*\ymin}) -- (\xmax,\ymin,{\cp+\xp*\xmax+\yp*\ymin}) -- (\xmax,\ymax,{\cp+\xp*\xmax+\yp*\ymax}) -- (\xmin,\ymax,{\cp+\xp*\xmin+\yp*\ymax}) -- (\xmin,\ymin,{\cp+\xp*\xmin+\yp*\ymin});

%Plot z = \cq + (\xq)*x + (\yq)*y
\draw[planecolor2!45!black, fill=planecolor2!45, fill opacity=0.25] (\xmin,\ymin,{\cq+\xq*\xmin+\yq*\ymin}) -- (\xmax,\ymin,{\cq+\xq*\xmax+\yq*\ymin}) -- (\xmax,\ymax,{\cq+\xq*\xmax+\yq*\ymax}) -- (\xmin,\ymax,{\cq+\xq*\xmin+\yq*\ymax}) -- (\xmin,\ymin,{\cq+\xq*\xmin+\yq*\ymin});

%Plot the intersection of the two planes as a dashed line.
\draw[planecolor1!50!planecolor2,thick,dashed] ({\xmin-\axisextra},{-1*(\xmin-\axisextra)*((\xp-\xq)/(\yp-\yq)) + (\cq-\cp)/(\yp-\yq)},{-1*(\xmin-\axisextra)*(\yp*(\xp-\xq)/(\yp-\yq)-\xp) + (\cp+\yp*(\cq-\cp)/(\yp-\yq))}) -- ({\xmax+\axisextra},{-1*(\xmax+\axisextra)*((\xp-\xq)/(\yp-\yq)) + (\cq-\cp)/(\yp-\yq)},{-1*(\xmax+\axisextra)*(\yp*(\xp-\xq)/(\yp-\yq)-\xp) + (\cp+\yp*(\cq-\cp)/(\yp-\yq))});
\end{tikzpicture}
\caption{An example of three dimensional graphing.}
\label{fig}
\end{figure}

\begin{figure}[ht]\centering
\subfloat[A quadratic through three points.]{%
\begin{tikzpicture}[scale=.75]
\def\xmin{-1}
\def\xmax{4}
\def\ymin{-3}
\def\ymax{3}
\def\xscale{1.5}
\def\yscale{1}
\def\axisextra{.5}

\draw[xstep=\xscale, ystep=\yscale, style = help lines, very thin] ({\xscale*\xmin-\axisextra/2},{\yscale*\ymin-\axisextra/2}) grid ({\xscale*\xmax+\axisextra/2},{\yscale*\ymax+\axisextra/2});
\foreach \x in {\xmin,...,-1} {\draw ({\x*\xscale},0) node[below, fill=white]{\tiny{\x}};}
\foreach \x in {1,...,\xmax} {\draw ({\x*\xscale},0) node[below, fill=white]{\tiny{\x}};}
\foreach \y in {\ymin,...,-1} {\draw (0,{\y*\yscale}) node[left, fill=white]{\tiny{\y}};}
\foreach \y in {1,...,\ymax} {\draw (0,{\y*\yscale}) node[left, fill=white]{\tiny{\y}};}
\draw[<->] ({\xscale*\xmin-\axisextra},0) -- ({\xscale*\xmax+\axisextra},0) node[right]{\tiny{x}};
\draw[<->] (0,{\yscale*\ymin-\axisextra}) -- (0,{\yscale*\ymax+\axisextra}) node[above]{\tiny{y}};
\begin{scope}
\clip ({\xscale*\xmin-\axisextra/2},{\yscale*\ymin-\axisextra/2}) rectangle ({\xscale*\xmax+\axisextra/2},{\yscale*\ymax+\axisextra/2});
\draw[thick] plot[domain={\xscale*\xmin-\axisextra/2}:{\xscale*\xmax+\axisextra/2}, samples=200]({\x*\xscale},{( -1*\x*\x + 4*\x - 2 )*\yscale});
\end{scope}

\def\therad{.08}
\draw[fill=black] ({\xscale*1},{\yscale*1}) circle(\therad) node[above left]{\small{$(1,1)$}};
\draw[fill=black] ({\xscale*2},{\yscale*2}) circle(\therad) node[above right]{\small{$(2,2)$}};
\draw[fill=black] ({\xscale*3},{\yscale*1}) circle(\therad) node[above right]{\small{$(3,1)$}};
\end{tikzpicture}
\label{fig:ThreePoints_Quadratic}}\qquad
\subfloat[Cubics through three points.]{%
\begin{tikzpicture}[scale=.75]
\def\xmin{-1}
\def\xmax{4}
\def\ymin{-3}
\def\ymax{3}
\def\xscale{1.5}
\def\yscale{1}
\def\axisextra{.5}

\draw[xstep=\xscale, ystep=\yscale, style = help lines, very thin] ({\xscale*\xmin-\axisextra/2},{\yscale*\ymin-\axisextra/2}) grid ({\xscale*\xmax+\axisextra/2},{\yscale*\ymax+\axisextra/2});
\foreach \x in {\xmin,...,-1} {\draw ({\x*\xscale},0) node[below, fill=white]{\tiny{\x}};}
\foreach \x in {1,...,\xmax} {\draw ({\x*\xscale},0) node[below, fill=white]{\tiny{\x}};}
\foreach \y in {\ymin,...,-1} {\draw (0,{\y*\yscale}) node[left, fill=white]{\tiny{\y}};}
\foreach \y in {1,...,\ymax} {\draw (0,{\y*\yscale}) node[left, fill=white]{\tiny{\y}};}
\draw[<->] ({\xscale*\xmin-\axisextra},0) -- ({\xscale*\xmax+\axisextra},0) node[right]{\tiny{x}};
\draw[<->] (0,{\yscale*\ymin-\axisextra}) -- (0,{\yscale*\ymax+\axisextra}) node[above]{\tiny{y}};
\begin{scope}
\clip ({\xscale*\xmin-\axisextra/2},{\yscale*\ymin-\axisextra/2}) rectangle ({\xscale*\xmax+\axisextra/2},{\yscale*\ymax+\axisextra/2});
\draw[thick,red] plot[domain={\xscale*\xmin-\axisextra/2}:{\xscale*\xmax+\axisextra/2}, samples=200]({\x*\xscale},{( -1*\x*\x*\x/2 + 2*\x*\x - 3*\x/2 + 1 )*\yscale});
\draw[thick,blue] plot[domain={\xscale*\xmin-\axisextra/2}:{\xscale*\xmax+\axisextra/2}, samples=200]({\x*\xscale},{( -\x*\x*\x/6 + 13*\x/6 - 1 )*\yscale});
\end{scope}

\def\therad{.08}
\draw[fill=black] ({\xscale*1},{\yscale*1}) circle(\therad) node[above left]{\small{$(1,1)$}};
\draw[fill=black] ({\xscale*2},{\yscale*2}) circle(\therad) node[above right]{\small{$(2,2)$}};
\draw[fill=black] ({\xscale*3},{\yscale*1}) circle(\therad) node[above right]{\small{$(3,1)$}};
\end{tikzpicture}
\label{fig:ThreePoints_Cubic}}
\caption{How to make graphs of functions and how to make subfigures.}
\label{fig:ThreePoints}
\end{figure}

\end{document}
